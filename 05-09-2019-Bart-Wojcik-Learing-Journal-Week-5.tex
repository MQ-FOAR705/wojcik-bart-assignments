\documentclass{article}
\usepackage[utf8]{inputenc}

\title{Week 5 - Learning Journal   - Bart Wojcik}
\author{Bart Wojcik}
\date{5 September 2019}

\begin{document}

\maketitle

\section{OpenRefine for Social Science Data}
\subsection{Setup}
\begin{enumerate}
    \item Downloaded and extracted OpenRefine 3.2.
    \item Attempted to run software. Found that OpenRefine depends on JRE. I was a bit grumpy about having to instsall the Java Runtimes as I am not a fan of polluting my system with garbage, one-off dependencies. My experience with VirtualBox (also an Oracle product) on Windows, is that the uninstaller never completely gets rid of all the non-user facing files.
    \item Downloaded the JRE and attempted to run again.
    \item Command prompt window opened and eventually a web interface launched in my default browser. I wasn't expecting this would run like a local server with a web interface, doesn't matter.
    \item Software executed successfully.
    \item Lesson concluded.
\end{enumerate}
\subsection{Introduction}
\begin{enumerate}
    \item Completed reading.
\end{enumerate}
\subsection{Working with OpenRefine}
\begin{enumerate}
    \item Successfully imported. Played a bit with the preview by changing the delimiter type to see what a bad import would look like. Reverted to default CSV to proceed with work.
    \item Created the project.
    \item Used a text facet on the village column as instructed. Found the following issues when sorting: inconsistent vocabulary use between 'Chirdozo' and 'Chirodzo', 'Ruaca' and 'Ruca', 'Ruaca - Nhamuenda' and 'Ruaca-Nhamuenda'. '49' does not appear to be the right data for this field.
    \item Completed exercise, found: 19 unique results for 'interview\verb|_|date', the column appears to be formated as text, transformed to date as instructed, had a look at the available facet and chose 'Timeline' in the way of testing, arrived at a timeline graph. Found that most interview data had been collected in 11/2016.
    \item Tested clustering as instructed, found no more clustering after using 'metaphone3'
    \item Manually narrowd down to four clusters as instructed.
    \item Moving to Transformingg data'. Removed all left square brackets as instructed. Removed single quotes using same method. Removed right square brackets.
    \item Split items owned using a custom text facet, noticed several duplicate labels.
    \item 'Mobile phone' and 'radio' appear to be the two most owned items; 'car' and 'computer' the least owned. The duplicate labels need clean-up before this step, why is this missing from the worksheet?!? I thought the idea was to get the machine to do the job not have me look for repeated instances of a label.
    \item November was the month most reported for food lack.
    \item Performed \newline \verb|"value.replace("[", "").replace("]", "").replace("'", "").replace(" ", "")"| on \verb|months_no_water|, \verb|liv_owned|, \verb|res_change|, \verb|no_food_mitigation| by recycling the command using the history tab.
    \item Scoped the undo\verb|/|redo feature.
    \item Trimmed leading and trailing whitespaces and ended with only four labels as predicted by the exercise.
    \item Concluded the exercise.
\end{enumerate}
\subsection{Filtering and Sorting with OpenRefine}
\begin{enumerate}
    \item Completed the filtering section, the selected types of roof were all the types that contained the string 'mabat', to further restrict the selection the user should provider a more closely phrased match.
    \item Scoped the \verb|include\exclude| feature.
    \item Sorting does nothing in my software, the command is simple and I have not made a mistake. I had to refresh the page for it to work again. Annoying, what a colossal waste of my time doing this is. To answer the questions of whether there is anything wrong, the lesson makes a tall assumption about my understanding of GPS based altitude recording, if I had to, I would say zero means 'sea level' but the lesson solution says that it means the device could not determine a value. 
    \item Completed the sorting exercise and quickly discovered by looking at GPS coordinates that village 49 should be Chirodzo.
    \item Changed the village name as instructed.
\end{enumerate}

\section{Elaboration Tests}
These correspond directly to item 4 in my Elaboration I document.
\begin{enumerate}
    \item Virtual Machine - I had no problems setting up Oracle VirtualBox on a Windows 10 host, I've included the Expansion Pack to get the enahnced support for modern hardware. I looked into alternatives like QEMU and the Windows Subsystem for Linux (WSL) but decided that in terms of setting up testing environments Oracle's VirtualBox provides the broadest compatibility and support for my guest operating systems. WSL for example, would limit me to working with Linux where I also required the ability to run MacOS on a Windows 10 host.
    \item Target Platforms - I am already running a Windows 10 machine which takes care of a Windows development and testing environment. \newline For Linux, I have initially thought of using Linux Arch because of the low system requirements but the set-up process requirements exceeded the amount of time available to me. I have ultimately decided on Ubuntu, a Debian derivative with further forks of its own (Kubuntu, Lubuntu, Mint) as the lowest available to me common denominator between contemporary Linux distributions. There is a good chance that if a Linux user gets their hands on the research environment that I am preparing, they will be an Ubuntu or another Debian derivative user. I have set-up Ubuntu as a guest system in VirtualBox and pre-configured a basic user environment. \newline MacOS was the tricky part but I was able to source a MacOS Mojave ISO. Initially the ISO would not boot and simply loop around on an unmodified Virtual Machine (VM). I looked up and used an online tutorial at https://www.howtogeek.com/289594/how-to-install-macos-sierra-in-virtualbox-on-windows-10/ to modify my Virtual Machine and make the MacOS GUI installer bootable. Since I already had an ISO (although for a different version) and knew how to configure my VM, I only needed to start at 'Step Four' of the tutorial to prepare the VM for a MacOS guest OS. The only changes I had to make to the 'Step Four' of the tutorial were to substitute the name of the Virtual Machine: the commands provided by the tutorial referred to a VM called "High Sierra" where on my system, I named my Mac VM "MacOS". After completing this part of the tutorial I was able to simply use the GUI installer, partition the virtual disk and install the system. I did not have to take any further steps and was able to boot into a working MacOS guest environment. Once in MacOS, I updated the entire system to the latest stable release and set up a typical user environment.
    \item Browser base - I was unable to yet look into how or where from to source an unbranded version of Firefox or how does the Tor Project pre-package their binaries for distribution. This is a scheduling issue not a technology issue however, and will be addressed by Week 8. Having said that, I have looked into other browsers like Edge and Chrome as possible hosts for my research environment but decided on Firefox rather than Chrome because of the projects commitment to user privacy. Edge and Internet Explorer did not offer the plugin support I required, there were no hover dictionaries available for these browsers, disqualifying both of them early on.
    \item Japanese - English hover dictionary. The original Rikaichan plugin for Firefox is now an abandoned project, its extended fork Rikaisama has also been abandoned since Firefox Quantum moved to a new plugin format, there are alternatives however. Rikaichamp is the most direct contemporary fork of Rikaichan and includes all the features which I require - word loop-up on mouse hover and de-inflection of conjugated parts of speech. There is an alternative in Yomichan, a newever plugin which additionally allows a power user to import their EPWING dictionaries, however Yomichan has proven sluggish in testing. Rikaichamp was fast responsive and worked out of the box with minimum set up whilst providing all the functionality I require.
    \item Annotation solution - Out of my two candidates I was only able to test Liner at this time. Liner offers very limited functionality, there are only two highlight colours available and the overall functionality is restricted for non-paying users. On the other hand, Hypothesis is free and open, empirical testing will still need to be conducted but at this time Hypothesis looks to be a good choice.
    \item Packaging and deployment. In Elaboration I, I intended to look at the Mozilla Institutional Deployment tool as a pre-configuration, packaging and deployment solution but reviewing the documentation even at a glance reveals a pre-occupation with MSI packages which are in fact specific to Windows. As I require a way of packaging a pre-configured distribution of Firefox for not just Windows but also MacOS and Linux, this disqualifies the Mozilla Institutional Deployment tool as a solution.
    \item Conclusion - Further testing and technology research still required.
\end{enumerate}
\end{document}