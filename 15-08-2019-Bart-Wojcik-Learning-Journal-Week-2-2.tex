\documentclass{article}
\usepackage[utf8]{inputenc}

\title{Week 2 - Learning Journal 2 - Bart Wojcik}
\author{Bart Wojcik}
\date{15 August 2019}

\begin{document}

\maketitle

\section{Data Carpentry - Introduction}
(https://datacarpentry.org/spreadsheets-socialsci/00-intro/index.html)
\subsection{How many people have used spreadsheets in their research?}
There is no need for spreadsheet use in my research as it is purely qualitative and does in involve the kind of input or output data that is either quantifiable or structure-able. I have used spreadsheets professionally however.
\subsection{How many people have accidentally done something that made them frustrated or sad?}
This is not a problem for me.\newline
\newline
\textbf{NOTES:} I had to chuckle at this statement: "In this lesson, we will assume that you are most likely using Excel as your primary spreadsheet program - there are other programs with similar functionality but Excel seems to be the most commonly used.". They are probably correct but it is a tall assumption to make.

\section{Data Carpentry - Formatting data tables in Spreadsheets:}
(https://datacarpentry.org/spreadsheets-socialsci/01-format-data/index.html)

\subsection{Problems with the Mozambique sheet:}
\subsubsection{Dwelling table:}
\begin{enumerate}
    \item 'Rooms' value for 'key\_id' 3 is a negative numeral. I don't imagine having a negative number of rooms but it is impossible to guess what this value is meant to represent.
    \item 'Rooms' value for 'key\_id' 10 is highlighted to indicate the dwelling includes a barn. The coloured highlight will not export in a plain text format such as CSV and the data will be lost. A better approach would include another column includes barn which could be populated by a boolean value true or false.
    \item The text data for roof type is inconsistent (mabati\_sloping vs mabatisloping).
\end{enumerate}
\subsubsection{Livestock table:}
\begin{enumerate}
    \item 'livestock\_owned\_and\_numbers' column includes too many variable of too many types. There are text strings which are sort-able alphabetically an integers that could be sorted numerically but not while they occupy the same column, with the numerical data at the front, the column is currently only sort-able by number. The data only accounts for the total number of heads as combined across all types of owned livestock. Each data-type in this column, should in fact have its own column (e.g. oxen, cows, goats and so forth, should heave their own column each which allows for more granular data keeping.
    \item The table seems to be missing 'key\_id' 10 entry.
\end{enumerate}
\subsubsection{Plots table:}
\begin{enumerate}
    \item Data for 'key\_id' 10 is null.
    \item Value for 'plots' in 'key\_id' 5 is null.
    \item Value for 'plots' in 'key\_id' 9 is a negative value. It is impossible to know what this means but the presence of it may or may not indicate its significance.
    \item Variable types in the 'water\_use' column are inconsistent. 'water\_use' seems to be a binary variable that would be best represented by a Boolean value rather than a string or an integer. Even as text string the data was not recorded consistently. To record seasonal water use data a column for each season taking a Boolean value may be a better choice.
    \subsection{Problems with the Tanzania sheet:}
    \subsubsection{Dwelling table:}
    \begin{enumerate}
        \item Same is as in the same table in the Mozambique sheet except the additional data is indicated by an asterisk rather than a highlight. Although the asterisk can be exported to a plain text format like CSV, it will be meaningless as all other values are simple integers, it will become garbage data. The solution is the same as in the Mozambique sheet.
    \end{enumerate}
    \subsubsection{Livestock table:}
    \begin{enumerate}
        \item Multiple cells contain null values, it is not clear whether this means that the reported value was zero or whether no data had been collected.
        \item Data is represented inconsistently with 'poultry', 'goats' and 'cows' columns corrupted with text strings between integers. For example, in 'key\_id' 5 this makes it necessary to assume that 'Yes' for 'Poultry' and 'Cows' equals 1 each, this is fine enough in this case however if the total number of livestock was, for argument's sake, five, the only knowable data would be the total number of livestock and the number of oxen.
    \end{enumerate}
    \textbf{NOTES:} In general, where binary value data is involved a Boolean value is more appropriate, where quantifiable data is involved, it should be represented by a numerical value. Where text string data is concerned, the usage should be consistent so that expect cases can be defined in machine language. Special cases should all have their own columns, even if there is only one special case per table, where the special case does not apply it may be denoted by a Boolean value 'false', for example 'cowsheds - true/false' for each 'key\_id'. To merge both sheets, the recording convention needs to be made consistent. For example as it is, it would impossible to merge the 'livestock' tables because each contains data recorded in a different manner.
\end{enumerate}
\subsubsection{Problems with the lack of metadata in the SAFI\_clean.csv spreadsheet:}
\begin{enumerate}
    \item What does 'key\_id' denote, is this an individual representing a household or a household itself or some other entity?
    \item What does 'village' actually denote, is this name or 'type' or 'configuration'. It is not immediately obvious what is being recorded.
    \item Why is the time being indicated even though it appears to have not been recorded, being the same for all 'key\_id' rows.
    \item What does 'years\_liv' indicated. The immediate guess is years lived but if 'key\_id' denotes a household and not an individual, does this value mean average lifespan or what else exactly.
    \item It is impossible to know what is meant by 'memb\_assoc' and how to interpret the associated three value 'NULL', 'yes', 'no'. The 'affect\_conflicts' column is equally vague.
    \item It can be extrapolated from the messy data that 'live\_count' means 'livestock head count' however this is not clear from the table.
    \item Explanation of how to treat 'NULL' data for the 'items\_owned' column is needed to interpret it.
    \item 'no\_meals' does it denote a number of meals or the number of instances were a meal was expected but not consumed. Whose meals?
    \item Explanation required for 'instanceID'.
\end{enumerate}
\section{Data Carpentry - Common Spreadsheet Errors:}
(https://datacarpentry.org/spreadsheets-socialsci/02-common-mistakes/index.html)
\begin{enumerate}
    \item Using multiple tables per sheet is a problem (to be fair, this depends on how the data is ultimately to be used, this has never proven to be a problem when I worked with retail and wholesale pricelists in my professional career, however in the context of this data it would create a garbage export to plain text).
    \item The point about using multiple tabs is fair, however I do not known the background of the data collection part of this project. It is conceivable (and how I imagined it) that the data was collected by multiple individuals and at the time the messy spreadsheet had been created, it had not been collated yet.
    \item Not filling in zeros and null values: I have made remarks about that.
    \item Using formatting to convey information: I have noted that.
    \item Using formatting to make the data sheet look pretty: I did not make a remark about merged cells for the same reason indicated in point 2, the data had not been collated and although this would render the spreadsheet problematic as an input file, I did not see it as being at that stage yet.
    \item Placing comments in data cells: I have remarked about that.
    \item Entering more than one piece of information in a cell: I have remarked about that.
    \item Using problematic field names: I have remarked about data naming consistency.
    \item Using special characters in data: I feel this falls into the formatting and note-taking-in-data-fields categories.
    \section{LaTeX Issues:}
    \begin{enumerate}
        \item First bullet point came out without an indent. Seems to be default behaviour at the beginning of a document. Added \\subsection and a sub section title. Added a line break below which did not appear, I expect a LaTeX equivalent of the HTML </br> or nbsp; pair exists. \\break did not work, apparently it creates a spread text (whatever it is called). /linebreak did the same thing, rather than guessing I Googled and found that /medskip was what I was after.
        \item Apparently the issue with the stretched text was due to my misspelling of \\subsection. I've fixed the indentation issue with \\parindent and added skips and \\newline commands between bullet points for readability. This concludes putting the 14/08/19 journal into LaTeX.
        \item  I received a message when submitting (it's unclear where it is submitting to, I did connect to GitHub but did not select a repo, the Overleaf UI is ambiguous here, apparently there is a LaTex error but when I hover over it the error is explained as 'underfull hbox badness 10000' followed by the paragraph and line number. Not very helpful, apparently the issue was cause by how I used \\newline.
        \item I've gone back to rebuild the bullet list using a native LaTeX syntax (I had pasted it across from OneNote initially causing it to be treated as plain text and not breaking lines which made me use \\newline etc etc). I got rid of the \\medskips and recompiled but then I had to go back and remove the plain-text bullets that came across from One Note.
        \item Lol, so apparently the 'submit' button sends your document to a Journal, poor UX they should label it better. I've had a look at the menu and found that I want to go to Menu -> Sync ->GitHub but why won't it see my repo, it only asks me to create a new one. Poor UX. I'll try making the rep public.
        \item Apparently inserting underscore in text mode requires the use of an escape character.
    \end{enumerate}
\end{enumerate}
\end{document}